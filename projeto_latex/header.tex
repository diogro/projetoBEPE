\usepackage[english]{babel} % suporte para l�nguas
\usepackage[utf8] {inputenc} % codifica��o
\usepackage{subfig, epsfig}
\captionsetup[subfigure]{style=default,
  margin=0pt, parskip=0pt, hangindent=0pt, indention=0pt,
  singlelinecheck=true, labelformat=parens, labelsep=space}
\usepackage{ae}
\usepackage{aecompl}
\usepackage{booktabs}
\usepackage[T1] {fontenc}
\usepackage{footnote}
% Notas criadas nas tabelas ficam no fim das tabelas
\makesavenoteenv{tabular}
\usepackage{fancyhdr}
% Usar os estilos do pacote fancyhdr
\fancypagestyle{plain}{
\fancyhf{}
\fancyfoot[C]{\thepage}
\renewcommand{\headrulewidth}{0pt}
\renewcommand{\footrulewidth}{0pt}
\headheight 13.6pt}
\usepackage{tabularx}
\usepackage{graphicx,wrapfig} % para incluir figuras
\usepackage[all]{xy} % para incluir diagramas
\usepackage{amsfonts, amssymb, amsthm, amsmath, amscd, textcomp} % pacote AMS
\usepackage{color, float, bbm, multicol, rotating}
\usepackage{verbatim, listings, booktabs}
\usepackage{caption} % Customizar as legendas de figuras e tabelas
\usepackage{array} % Elementos extras para formata��o de tabelas
\usepackage{lineno} % números nas linhas
\usepackage {tocvsec2} % controlar profundidade de table of contents
\setcounter {secnumdepth}{0}
\setcounter {tocdepth}{2}
%\widowpenalty10000
%\clubpenalty10000
\usepackage{mathpazo} % fonte palatino

% Adicionar bibliografia, �ndice e conte�do na Tabela de conte�do
% Não inclui lista de tabelas e figuras no índice
\usepackage[nottoc,notlof,notlot, notindex]{tocbibind}
\usepackage{icomma} % Posicionar inteligentemente a v�rgula como separador decimal
\usepackage[tight]{units} % Formatar as unidades com as dist�ncias corretas
\usepackage{setspace}
\onehalfspacing
\usepackage{lastpage} % Conta o n�mero de p�ginas
\usepackage[hmargin = 2.5 cm,vmargin = 2.5 cm]{geometry}
\usepackage{pdflscape} % ambiente landscape
%\geometry{bindingoffset=10pt}
\usepackage {tabularx}
\usepackage[round]{natbib}
\usepackage{chapterbib}
% --- defini��es gerais ---
%\newcommand{\barra}{\backslash}
%\newcommand{\To}{\longrightarrow}
%\newcommand{\abs}[1]{\left\vert#1\right\vert}
%\newcommand{\set}[1]{\left\{#1\right\}}
%\newcommand{\seq}[1]{\left<#1\right>}
%\newcommand{\norma}[1]{\left\Vert#1\right\Vert}
\newcommand{\hr}{\par\noindent\hrulefill\par}
% --- ---
\usepackage{longtable}

