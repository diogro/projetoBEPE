\documentclass[twoside,a4paper,12pt]{article}

\usepackage[english]{babel} % suporte para l�nguas
\usepackage[utf8] {inputenc} % codifica��o
\usepackage{subfig, epsfig}
\captionsetup[subfigure]{style=default,
  margin=0pt, parskip=0pt, hangindent=0pt, indention=0pt,
  singlelinecheck=true, labelformat=parens, labelsep=space}
\usepackage{ae}
\usepackage{aecompl}
\usepackage{booktabs}
\usepackage[T1] {fontenc}
\usepackage{footnote}
% Notas criadas nas tabelas ficam no fim das tabelas
\makesavenoteenv{tabular}
\usepackage{fancyhdr}
% Usar os estilos do pacote fancyhdr
\fancypagestyle{plain}{
\fancyhf{}
\fancyfoot[C]{\thepage}
\renewcommand{\headrulewidth}{0pt}
\renewcommand{\footrulewidth}{0pt}
\headheight 13.6pt}
\usepackage{tabularx}
\usepackage{graphicx,wrapfig} % para incluir figuras
\usepackage[all]{xy} % para incluir diagramas
\usepackage{amsfonts, amssymb, amsthm, amsmath, amscd, textcomp} % pacote AMS
\usepackage{color, float, bbm, multicol, rotating}
\usepackage{verbatim, listings, booktabs}
\usepackage{caption} % Customizar as legendas de figuras e tabelas
\usepackage{array} % Elementos extras para formata��o de tabelas
\usepackage{lineno} % números nas linhas
\usepackage {tocvsec2} % controlar profundidade de table of contents
\setcounter {secnumdepth}{0}
\setcounter {tocdepth}{2}
%\widowpenalty10000
%\clubpenalty10000
\usepackage{mathpazo} % fonte palatino

% Adicionar bibliografia, �ndice e conte�do na Tabela de conte�do
% Não inclui lista de tabelas e figuras no índice
\usepackage[nottoc,notlof,notlot, notindex]{tocbibind}
\usepackage{icomma} % Posicionar inteligentemente a v�rgula como separador decimal
\usepackage[tight]{units} % Formatar as unidades com as dist�ncias corretas
\usepackage{setspace}
\onehalfspacing
\usepackage{lastpage} % Conta o n�mero de p�ginas
\usepackage[hmargin = 2.5 cm,vmargin = 2.5 cm]{geometry}
\usepackage{pdflscape} % ambiente landscape
%\geometry{bindingoffset=10pt}
\usepackage {tabularx}
\usepackage[round]{natbib}
\usepackage{chapterbib}
% --- defini��es gerais ---
%\newcommand{\barra}{\backslash}
%\newcommand{\To}{\longrightarrow}
%\newcommand{\abs}[1]{\left\vert#1\right\vert}
%\newcommand{\set}[1]{\left\{#1\right\}}
%\newcommand{\seq}[1]{\left<#1\right>}
%\newcommand{\norma}[1]{\left\Vert#1\right\Vert}
\newcommand{\hr}{\par\noindent\hrulefill\par}
% --- ---
\usepackage{longtable}



% Links dinâmicos
\usepackage{hyperref}
\hypersetup{colorlinks=true, linkcolor=black, citecolor=black, filecolor=black, pagecolor=black, urlcolor=black,
        pdfauthor={Diogo Melo},
        pdftitle={The Evolution of Pleiotropy},
        pdfsubject={Biologia Evolutiva},
        pdfkeywords={Evolução morfológica, Pleiotropia, QTL},
        pdfproducer={Latex},
        pdfcreator={pdflatex}}
% Pontuação e unidades
\usepackage{icomma}

% Capa

% Se quiser uma figura de fundo na capa ative o pacote wallpaper
% e descomente a linha abaixo.
% \ThisCenterWallPaper{0.8}{nomedafigura}


%\renewcommand*\familydefault{\sfdefault}

%\usepackage{pslatex}
% Início do texto
\begin{document}
\pagestyle{empty}
\begin{titlepage}

\begin{center}
\large{Projeto de Pesquisa-Bolsa de Estágio no Exterior (BEPE)-FAPESP}\\

%\topskip0pt
\vspace*{\fill}

\LARGE{Estimativas Diretas de Parâmetros Evolutivos Via Análise de QTL}\\

\vspace{20pt}
\small{Associado ao processo de N$^{o}$ 2014/26262-4}
\vspace{2cm}


\vspace*{\fill}

\small{\emph{Aluno:} Diogo Amaral R. \textsc{Melo}\\
\emph{Orientador:} Gabriel Henrique \textsc{Marroig}} Zambonato\\
\emph{Supervisor:} Jason \textsc{Wolf}\\
\vspace{10pt}
\large{Laboratório de Evolução de Mamíferos}
\\
\vspace{10pt}
\large{\today}

\end{center}

\thispagestyle{empty}
\end{titlepage}



\noindent
\large{\emph{Title:} Direct Estimates of Evolutionary Parameters via QTL Analysis}\\
\begin{normalsize}
\noindent
\emph{Student:} Diogo Amaral R. \textsc{Melo}\\
\emph{Advisor:} Gabriel Henrique \textsc{Marroig} Zambonato\\
\emph{Supervisor:} Jason \textsc{Wolf}\\
\emph{To be developed at:} University of Bath (Bath), UK\\
\end{normalsize}

\begin{abstract}

Como a evolução morfológica e a resposta à seleção dependem da
variação disponível em uma dada população, como essa estrutura de
covariação evolui é uma pergunta fundamental em biologia evolutiva.
A estrutura de covariação genética é determinada pelo conjunto
de efeitos genéticos, incluindo principalmente genes pleiotrópicos
e desequilíbrio de ligação. Neste estágio pretendemos utilizar
dados de cruzamentos entre linhagens de camundongos com tamanhos
altamente divergentes para estudar a relação entre efeitos genéticos
pleiotrópicos e a variação genética encontrada nessas linhagens.
Utilizando técnicas de mapeamento de QTL e medidas morfológicas,
podemos usar esse sistema experimental poderoso para compreender a
evolução da covariação e sua base genética.


\textbf{Keywords}: Morphological evolution, Pleiotropy, Micro-evolution, G Matrix, Quantitative genetics
\end{abstract}

\tableofcontents

\thispagestyle{empty}
\newpage

\pagestyle{plain}
\onehalfspacing
\setcounter{page}{1}
\pagenumbering{arabic}

\section{Introduction}

Genética quantitativa

Mapa Genótipo-fenótipo

Variação

QTL

\section{Objectives of the Internship}

In this project we intend to use available data on the cross
between two inbred mice strains LG/J ({\it Large\/}) and SM/J
({\it Small\/}) to study how pleiotropic effects on multivariate
phenotypes evolve under selection. Parental individuals were obtained
from mice strains independently selected for increase and decrease
in size~\citep{MacArthur1944, Goodale1938}. We have access to data
phenotypic and genotypic data from around 1500 individuals from the
$F_3$ generation, distributed among 195 families. For all these
individuals, we have measurements on 3 sets of phenotypes: weekly weights
for the first eight weeks after birth, morphological landmarks in the
jaw and several internal organ weight at death.

All sets of multivariate phenotypes have similar dimensionality, that
is, they are composed by a similar number of traits. Also, all phenotype
sets are divergent between the parental strains, and so are informative
on how directional selection affected their genetic structure. Given
this, we intend to use these complementary phenotypes to understand
how genetic effects evolve, focusing especially how pleiotropy changes
under selection and how this affects the patterns of standing genetic
covariation. In order to achieve this goal, we intend to map the QTLs
associated with each phenotypic trait, and to estimate vectors of
pleiotropic additive and dominance effects for each marker. Then, we will
use these effects vectors to elucidate how selection on size altered
the genetic architecture in the selected strains, with special focus on
changes is pleiotropy.

Using these additive and dominance effects vectors, we can also
investigate how genetic correlation between traits changed under
selection on size, and how these changes altered the evolutionary
potential of the selected strains. Also, we can use quantitative
genetics to directly estimate how different genetic effects (like
additive and dominance) contribute to the genetic covariance matrix, and
how selection changed this standing genetic covariation.

In order to complete this project, I intend to learn QTL mapping
techniques, along with the relevant quantitative genetics theory,
which related the effects of the alleles with standing variation
and response to selection. As the data we plan on using has already
been collected, we believe this internship will be crucial in providing
me and the Mammal Evolution Laboratory in Brazil the theoretical and
computational tools to answer new and existing questions in evolutionary
biology and genetics, furthering a leading research program at the
interface between genetics and evolution that is at this time absent in
Brazil. In this sense, the supervisor Jason Wolf is an ideal partner,
having vast experience in QTL mapping and quantitative genetics. He is
also lead author of tens of articles in these topics, both theoretical
and experimental, in high impact journals. Furthermore, this internship
fits naturally with the main theme of my doctorate, and allows developing
the necessary tools for its completion.

\section{Project Schedule}

The project will be developed in the first semester of 2016, during a 6
month period (Table~\ref {tab:crono}). During the first three months I
will dedicate myself to learning marker mapping methods and the relevant
quantitative genetics theory. During this period I also plan to develop
any software that proves necessary for the analysis of multivariate
pleiotropic effects. In the latter months, I intend to finish
analyzing the available data and to begin work on at least one manuscript.

\begin{table}[H]
    \hr
    \caption {Project work schedule.}
    \hr
    \begin{center}
    \begin{tabular}{lc*{4}{cc}}
        \toprule
    Months & $1^{th}$ & $2^{th}$ & $3^{th}$ & $4^{th}$ & $5^{th}$ & $6^{th}$ \\
              \midrule
  Literature review &$\blacksquare$&$\blacksquare$&$\blacksquare$&$\blacksquare$& $\blacksquare$&$\blacksquare$\\
  Implementation &$\blacksquare$&$\blacksquare$&$\blacksquare$&-&-&-\\
  Data analysis &-&-&$\blacksquare$&$\blacksquare$&$\blacksquare$&$\blacksquare$\\
  Writing  &-&-&-&-&$\blacksquare$&$\blacksquare$\\
  \bottomrule

  \end{tabular}
  \end{center}
  \label{tab:crono}
\end{table}

\bibliographystyle{apalike}
\singlespacing
\begin{multicols} {2}
\scriptsize{\bibliography{bibliography}}
\end{multicols}
\end{document}
